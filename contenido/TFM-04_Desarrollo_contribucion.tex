\chapter{Desarrollo específico de la contribución}\label{chap:desarrollocontri}
En este bloque debes detallar el desarrollo de tu contribución. A continuación, te presentamos la estructura habitual para el tipo de trabajo considerado en este Máster. Tipo general: Desarrollo práctico. En este tipo de trabajo de desarrollo práctico, es importante justificar los criterios de planificación, análisis y diseño empleados para desarrollar el software, seguido de la descripción detallada del producto resultante y finalizando con una evaluación de la calidad y aplicabilidad del producto. Esto suele verse reflejado en la siguiente estructura de subapartados.
\section{Planificación / Análisis / Requisitos}\label{sec:plani}
En este apartado se debe indicar el trabajo previo realizado para guiar el desarrollo del software. Esto debería incluir la identificación adecuada del problema a tratar, así como del contexto habitual de uso (empresa, institución, etc.), relacionado y complementando con lo que se haya podido incluir en los capítulos de introducción y contexto y estado del arte. 
La tarea de planificación considerará todos los detalles a considerar para el desarrollo y el proceso. Idealmente, el análisis y la identificación de requisitos se debería hacer contando con expertos en la materia a tratar, en la medida de lo posible. Además, deberás describir en detalle las características del sistema. Como mínimo querrás mencionar:
	Qué tecnologías se utilizaron (incluyendo justificación de por qué se emplearon y descripciones detalladas de las mismas).
	Cómo se organizó el desarrollo.
	Qué personas participaron (con datos demográficos, si procede) o qué técnicas de sistemas se emplearon.
	Cómo transcurrió el desarrollo.
	Qué instrumentos de seguimiento y evaluación se utilizaron durante el proceso de desarrollo.

\section{Descripción del sistema desarrollado / Implementación}\label{sec:descripcionsistema}
Se deberán aportar detalles del proceso de desarrollo, incluyendo las fases e hitos del proceso.  También deben presentarse diagramas explicativos de la arquitectura o funcionamiento, así como capturas de pantalla que permitan al lector entender el funcionamiento del programa. Igualmente, y muy importante, detallar las tecnologías que se han empleado (se puede hablar más sobre esto en el capítulo de contexto) y cómo se han empleado (sería lo principal a incluir en este capítulo de desarrollo), el uso de la metodología apropiada usada, así como las decisiones tomadas durante el proceso y la implementación llevada a cabo. Se pueden también incluir algunas cuestiones que se consideren interesantes para el entendimiento del desarrollo (cuando sean muy extensas se pueden llevar a los Anexos).
\section{Evaluación}\label{sec:evaluacion}
La evaluación debería cubrir al menos una mínima evaluación de la usabilidad de la herramienta, así como de su aplicabilidad para resolver el problema propuesto. Estas evaluaciones suelen realizarse con usuarios expertos, de ser posible, o con pruebas de usabilidad más básicas. 
En cualquier caso, sería de especial relevancia y valor añadido el incluir también pruebas unitarias, de integración, según lo que corresponda, validando así la propuesta en la medida de lo posible. Esto es algo crucial para que queden más clarificadas las contribuciones que aporta el desarrollo.

\subsection{subseccion}\label{sec:subseccion}
cosas

