\chapter{Desarrollo específico de la contribución}\label{chap:desarrollo}

\section{Planificación, Análisis y Requisitos}

\subsection{Contexto y problema}

En arquitecturas de microservicios, uno de los retos más importantes es lograr una observabilidad efectiva. Cada servicio se ejecuta de forma independiente, interactúa con distintos componentes y genera información distribuida que, de no gestionarse correctamente, dificulta la detección de fallos, el diagnóstico y la mejora continua del sistema. La complejidad aumenta cuando se busca correlacionar información entre servicios, identificar patrones de comportamiento o anticipar problemas.

Por este motivo, el objetivo del proyecto fue desarrollar un prototipo de plataforma de observabilidad que integre métricas, logs y trazabilidad de manera sencilla, permitiendo a los desarrolladores y equipos de operaciones tener visibilidad sobre el funcionamiento de los microservicios, sin depender de herramientas complejas ni entornos productivos.

\subsection{Justificación de tecnologías}

Se eligieron herramientas maduras, con buena documentación y soporte comunitario, que permitieran un despliegue ágil y funcional:

\begin{itemize}
	\item \textbf{Prometheus}: captura de métricas de servicios y componentes, con compatibilidad con endpoints estándar de instrumentación.
	\item \textbf{Grafana}: visualización de métricas mediante dashboards configurables que permiten un seguimiento sencillo de la actividad de los servicios.
	\item \textbf{Jaeger y OpenTelemetry}: trazabilidad distribuida de solicitudes, lo que facilita comprender cómo las peticiones fluyen a través de los servicios.
	\item \textbf{Docker y Minikube}: contenerización y simulación de un clúster local de Kubernetes, permitiendo reproducir el entorno en cualquier máquina de desarrollo.
\end{itemize}

Se decidió dejar fuera ELK Stack en esta primera versión para reducir la complejidad, dejando abierta su integración en fases posteriores.

\subsection{Organización del desarrollo}

El trabajo se desarrolló de manera individual con planificación semanal, control de versiones mediante GitHub y documentación progresiva de cada fase. Esto permitió:

\begin{itemize}
	\item Registrar cada cambio en configuraciones y scripts.
	\item Garantizar la reproducibilidad del entorno de desarrollo.
	\item Documentar buenas prácticas de instrumentación de microservicios para futuros desarrollos.
\end{itemize}

\subsection{Requisitos funcionales y no funcionales}

\textbf{Funcionales:}
\begin{itemize}
	\item Captura de métricas básicas de los microservicios.
	\item Visualización de métricas mediante dashboards.
	\item Recolección de trazas de solicitudes entre servicios.
	\item Despliegue automatizado en un entorno local reproducible.
\end{itemize}

\textbf{No funcionales:}
\begin{itemize}
	\item Facilidad de replicación del entorno en otras máquinas.
	\item Documentación clara para replicar el prototipo.
	\item Ligereza del entorno, compatible con recursos limitados de desarrollo local.
\end{itemize}

\section{Descripción del sistema desarrollado e implementación}

\subsection{Arquitectura y diseño}

El prototipo integra tres microservicios Spring Boot instrumentados con OpenTelemetry y métricas compatibles con Prometheus. La arquitectura contempla:

\begin{itemize}
	\item Microservicios REST básicos que permiten simular flujos de solicitudes.
	\item Prometheus para recolección de métricas de CPU, memoria, tiempo de respuesta y número de peticiones.
	\item Grafana para visualización de métricas mediante dashboards configurables.
	\item Jaeger para trazabilidad de solicitudes distribuidas.
\end{itemize}

\begin{figure}[H]
	\centering
	\includegraphics[width=0.8\textwidth]{contenido/imagenes/arquitectura-tfm.png}
	\caption{Arquitectura simplificada del prototipo en entorno local.}
	\label{fig:arquitectura}
\end{figure}

\subsection{Implementación detallada}

\paragraph{Contenerización con Docker}

Cada microservicio y herramienta se encapsuló en contenedores Docker. Esto permitió:

\begin{itemize}
	\item Aislar dependencias de cada servicio.
	\item Reproducir el entorno fácilmente en cualquier máquina.
	\item Garantizar consistencia entre despliegues.
\end{itemize}

\paragraph{Despliegue en Kubernetes (Minikube)}

Se configuró un clúster local con Minikube para simular un entorno de producción:

\begin{itemize}
	\item Cada servicio desplegado mediante \texttt{Deployment} y \texttt{Service}.
	\item Volúmenes persistentes para almacenar logs y datos temporales.
	\item Configuración básica de Prometheus, Grafana y Jaeger para integrarlos con los microservicios.
\end{itemize}

\paragraph{Integración de observabilidad}

\begin{itemize}
	\item Prometheus recolecta métricas desde los microservicios.
	\item Grafana se conecta a Prometheus para generar dashboards básicos.
	\item Jaeger permite visualizar trazas de solicitudes que atraviesan múltiples servicios, mostrando tiempos de respuesta y secuencia de llamadas.
\end{itemize}

\subsection{Metodología de desarrollo}

Se siguió un enfoque iterativo, con entregas semanales que incluían desarrollo, despliegue y documentación. Esto permitió:

\begin{itemize}
	\item Validar la correcta instrumentación de los microservicios.
	\item Ajustar configuraciones y dashboards según necesidades.
	\item Documentar procedimientos de despliegue y operación del prototipo.
\end{itemize}

\section{Evaluación y pruebas}

\subsection{Pruebas funcionales}

Para validar la funcionalidad de la plataforma, se realizaron pruebas de monitoreo y trazabilidad:

\begin{itemize}
	\item Generación de tráfico simulado entre microservicios.
	\item Verificación de recolección de métricas en Prometheus.
	\item Visualización de métricas en dashboards de Grafana.
	\item Seguimiento de trazas completas en Jaeger, confirmando la correcta correlación de llamadas entre servicios.
\end{itemize}

\subsection{Usabilidad y aplicabilidad}

El prototipo permitió comprobar que:

\begin{itemize}
	\item Los microservicios están correctamente instrumentados para observabilidad básica.
	\item La recolección de métricas y trazas funciona de manera consistente en un entorno local.
	\item Se dispone de una base funcional para futuras ampliaciones hacia entornos productivos o integración de nuevas herramientas.
\end{itemize}

\subsection{Limitaciones y futuras mejoras}

\textbf{Limitaciones actuales:}
\begin{itemize}
	\item Alcance limitado a un prototipo local y microservicios simples.
	\item Funcionalidad de observabilidad básica, sin integración de alertas ni logs centralizados.
\end{itemize}

\textbf{Futuras mejoras:}
\begin{itemize}
	\item Integración de ELK Stack para gestión de logs centralizada.
	\item Automatización avanzada de despliegue con Helm y Ansible.
	\item Escalabilidad hacia entornos de nube o multi-nube.
	\item Creación de dashboards más completos con métricas personalizadas.
\end{itemize}
