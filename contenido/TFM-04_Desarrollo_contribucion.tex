\chapter{Desarrollo específico de la contribución}

\section{Planificación, Análisis y Requisitos}

El desarrollo de esta solución de observabilidad se basa en la necesidad de mejorar el monitoreo y trazabilidad en arquitecturas de microservicios, abordando un problema común en sistemas distribuidos.

\subsection{Contexto y problema}

Se identificó que en entornos de microservicios es complejo realizar un monitoreo integrado y efectivo que permita detectar fallos y analizar el comportamiento del sistema. Por ello, se planteó desarrollar una plataforma básica de observabilidad que integre métricas, logs y trazabilidad en un entorno local simulado.

\subsection{Justificación de tecnologías}

Dada la limitación de tiempo, se seleccionaron herramientas consolidadas y con amplia documentación que permitan un despliegue rápido y funcional:

\begin{itemize}
	\item \textbf{Prometheus y Grafana}: para la captura y visualización de métricas de forma sencilla.
	\item \textbf{Jaeger y OpenTelemetry}: para implementar trazabilidad básica distribuida.
	\item \textbf{Docker y Minikube}: para contenerización y despliegue local en un clúster Kubernetes simulado.
\end{itemize}

Se optó por no incluir ELK Stack en esta fase inicial para evitar complejidad adicional, dejando abierta su integración para trabajos futuros.

\subsection{Organización del desarrollo}

El trabajo se realizó de manera individual con una planificación semanal, utilizando control de versiones (Git) y documentación progresiva para garantizar la trazabilidad del proyecto.

\subsection{Requisitos funcionales y no funcionales}

\begin{itemize}
	\item \textbf{Funcionales:}
	\begin{itemize}
		\item Monitorización básica de métricas expuestas por microservicios.
		\item Visualización de métricas mediante dashboards en Grafana.
		\item Captura y visualización de trazas de solicitudes mediante Jaeger.
		\item Despliegue automatizado en entorno local con Minikube.
	\end{itemize}
	
	\item \textbf{No funcionales:}
	\begin{itemize}
		\item Reproducibilidad del entorno en distintas máquinas locales.
		\item Documentación clara para facilitar su uso y extensión.
	\end{itemize}
\end{itemize}

\section{Descripción del sistema desarrollado e Implementación}

El sistema desarrollado es una prueba de concepto que integra microservicios simples instrumentados para emitir métricas y trazas, desplegados en un clúster Kubernetes local mediante Minikube.

\subsection{Arquitectura y diseño}

La arquitectura consta de tres microservicios básicos desarrollados con Spring Boot, cada uno instrumentado con OpenTelemetry para la emisión de trazas, y métricas compatibles con Prometheus.

\begin{figure}[H]
	\centering
	\includegraphics[width=0.8\textwidth]{contenido/imagenes/arquitectura-tfm.png}
	\caption{Arquitectura simplificada para el entorno de desarrollo local.}
	\label{fig:arquitectura}
\end{figure}

\subsection{Implementación}

Cada microservicio se dockerizó y se desplegó en el clúster Minikube utilizando manifiestos Kubernetes básicos. Se instalaron Prometheus y Grafana con configuraciones mínimas para captar las métricas expuestas, y Jaeger para la trazabilidad distribuida.

Se configuraron dashboards simples en Grafana y visualizaciones básicas en Jaeger para demostrar el flujo de las solicitudes.

\subsection{Metodología y decisiones técnicas}

La metodología fue iterativa con entregas semanales que incluían desarrollo, pruebas y documentación. Se priorizó la funcionalidad mínima viable que permitiera demostrar el valor del sistema.

\section{Evaluación}

\subsection{Pruebas realizadas}

Se realizaron pruebas funcionales para validar la recolección de métricas y trazas, incluyendo la simulación de llamadas entre microservicios para verificar la trazabilidad.

\subsection{Usabilidad y aplicabilidad}

El sistema demostró ser una base funcional para entender cómo integrar observabilidad en microservicios, con una configuración reproducible en cualquier entorno local con Minikube.

\subsection{Limitaciones y futuras mejoras}

El alcance limitado a entorno local y funcionalidades básicas es una restricción conocida. Futuras mejoras incluyen:

\begin{itemize}
	\item Integración de gestión de logs con ELK Stack.
	\item Automatización avanzada del despliegue con Helm o Ansible.
	\item Despliegue en nube pública para validar escalabilidad.
\end{itemize}
