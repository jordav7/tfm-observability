\chapter{Objetivos y metodología de trabajo}\label{chap:estadodelarte}
El presente capítulo establece los objetivos que guían el desarrollo de este trabajo, tanto a nivel general como en aspectos específicos. Además, se describe la metodología adoptada para alcanzar dichos objetivos, detallando las fases, técnicas y herramientas que se utilizarán para implementar y evaluar la plataforma de observabilidad en microservicios con enfoque DevOps.

\section{Objetivo general}\label{sec:objgeneral}
El objetivo general de este trabajo es diseñar, implementar y evaluar una plataforma integral de observabilidad para arquitecturas de microservicios basada en prácticas DevOps, que permita el monitoreo, registro y trazabilidad efectivos de sistemas distribuidos, con el fin de mejorar la detección, diagnóstico y resolución de fallos, optimizando así la calidad y la fiabilidad del software.

\section{Objetivos específicos}\label{sec:objespecificos}
Para lograr el objetivo general, se plantean los siguientes objetivos específicos:

\begin{itemize}
	\item Definir y seleccionar claramente las herramientas de observabilidad que formarán parte del sistema, incluyendo Prometheus, Grafana, ELK Stack, OpenTelemetry y Jaeger.
	
	\item Elegir y configurar un repositorio de control de versiones y gestión del proyecto, optando por \textbf{GitHub} como plataforma base para el desarrollo colaborativo, así como utilizar \textbf{GitHub Actions} como herramienta de integración continua para automatizar los flujos de trabajo.
	
	\item Diseñar y desplegar una arquitectura de microservicios con Spring Boot que permita la integración de las herramientas de observabilidad.
	
	\item Implementar la automatización del despliegue utilizando Docker y Kubernetes, inicialmente sobre un entorno local (Minikube), con la opción de evaluar la incorporación de Terraform para la gestión de infraestructura en caso de ser necesario.
	
	\item Configurar el despliegue automatizado de las herramientas de observabilidad usando Helm y Ansible para facilitar la reproducibilidad y mantenimiento.
	
	\item Evaluar la eficacia del sistema para monitorizar, registrar y trazar las actividades y fallos en el entorno distribuido.
	
	\item Documentar exhaustivamente el proceso, incluyendo la configuración detallada de las herramientas, scripts de automatización y guías para replicar el entorno en otros contextos.
\end{itemize}

Cabe destacar que la selección de las herramientas utilizadas para cada uno de los pilares de la observabilidad (monitorización, logging y trazabilidad) ha sido detalladamente justificada en la Sección~\ref{sec:justificacion-eleccion-herramientas}. Dicha elección constituye la base sobre la cual se diseñará e implementará la solución propuesta en este trabajo, asegurando coherencia con los objetivos planteados y viabilidad técnica en entornos reales.


\section{Metodología del trabajo}\label{sec:metodologia_trabajo}
Para alcanzar los objetivos planteados, se seguirá la siguiente metodología:

\begin{enumerate}
	\item \textbf{Definición y selección de herramientas}: Confirmación de las tecnologías que se utilizarán, priorizando herramientas estándar y ampliamente soportadas como Prometheus, Grafana, ELK Stack, OpenTelemetry y Jaeger. Se utilizará \textbf{GitHub} como repositorio principal para la gestión del código fuente, y \textbf{GitHub Actions} como plataforma de integración continua para automatizar los flujos de trabajo de construcción, pruebas y despliegue.
	
	\item \textbf{Diseño de la arquitectura}: Construcción de un sistema de microservicios básico usando Spring Boot, que permita la integración de los componentes de observabilidad.
	
	\item \textbf{Implementación y despliegue inicial}: Contenerización de microservicios y herramientas, con despliegue automatizado en un entorno local mediante Minikube y Kubernetes.
	
	\item \textbf{Automatización con Helm y Ansible}: Creación y configuración de Helm charts para las herramientas de observabilidad, junto con playbooks de Ansible para facilitar la instalación y configuración repetible.
	
	\item \textbf{Evaluación y pruebas}: Realización de pruebas funcionales y de carga para verificar el correcto monitoreo, logging y tracing del sistema, generando métricas y trazas que permitan detectar fallos y cuellos de botella.
	
	\item \textbf{Documentación completa}: Registro detallado de cada paso, configuración y script utilizado, así como instrucciones claras para replicar el sistema, asegurando que otros equipos puedan utilizar esta solución de forma sencilla.
	
	\item\textbf{Posible extensión:} El despliegue se realizará sobre un entorno local, manteniendo abierta la opción de incorporar Terraform para la provisión automatizada de infraestructura en la nube, según se requiera.
\end{enumerate}


Esta metodología seguirá un enfoque iterativo y ágil, permitiendo ajustes continuos basados en resultados parciales y feedback.