\chapter{Conclusiones y trabajo futuro}\label{chap:resultados}

\section{Conclusiones}

Este estudio se ocupó del desafío de establecer una arquitectura de monitoreo para sistemas que emplean microservicios, centrándose en cómo reunir y representar métricas, trazas y registros mediante tecnologías de código abierto como OpenTelemetry, Prometheus y Grafana. Para probar la solución, configuramos un entorno local usando Kubernetes (Minikube), lo que nos permitió integrar y evaluar los elementos esenciales para el monitoreo y análisis del sistema.

A lo largo del proceso, diseñamos la arquitectura, equipamos los microservicios con el SDK de OpenTelemetry y ajustamos los componentes de monitoreo, incluyendo el OpenTelemetry Collector, Prometheus y Grafana. El resultado fue un entorno operativo que proporciona visibilidad en tiempo real sobre el rendimiento y rastreo del sistema, lo que ayuda en la identificación de cuellos de botella o fallas potenciales.

Los objetivos establecidos inicialmente, como construir un entorno reproducible para el monitoreo y demostrar su efectividad en local, se lograron con éxito. Además, la solución es lo suficientemente adaptable para futuras expansiones, tales como la migración a entornos en la nube o la automatización de procesos utilizando herramientas como Terraform y Kubernetes.

\section{Líneas de trabajo futuro}

A pesar de que el entorno actual satisface las expectativas, existen diversas áreas donde se puede continuar avanzando para incrementar el valor del proyecto:

\begin{itemize}
	\item \textbf{Implementación en la nube:} Automatizar la infraestructura en plataformas de nube pública utilizando Terraform y Kubernetes administrados, asegurando así escalabilidad y disponibilidad elevada.
	
	\item \textbf{Automatización avanzada:} Integrar pipelines de CI/CD que incluyan pruebas automatizadas, implementación y monitoreo, así optimizando la eficiencia del ciclo de desarrollo.
	
	\item \textbf{Seguridad y cumplimiento:} Establecer medidas de seguridad para el monitoreo, como el cifrado de datos tanto en tránsito como en reposo, además de controlar el acceso a paneles de control y datos sensibles.
	
	\item \textbf{Análisis avanzado:} Incorporar sistemas de análisis predictivo y alertas basadas en aprendizaje automático para prever fallos o deterioro del sistema.
	
	\item \textbf{Documentación y capacitación:} Preparar documentación detallada y recursos educativos para facilitar la adaptación por parte de los equipos de desarrollo y operaciones.
\end{itemize}

Estas líneas abren la oportunidad para una evolución natural de la plataforma de monitoreo, consolidando y ampliando el valor del trabajo realizado en este TFM.
