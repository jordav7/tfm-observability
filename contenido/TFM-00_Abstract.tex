\begin{abstract}[Resumen]
Resumen
\par
\par\vspace{0.25cm}
En las arquitecturas modernas basadas en microservicios, la complejidad y la naturaleza distribuida de los sistemas dificultan la detección, análisis y resolución eficiente de errores. Las soluciones tradicionales de monitoreo resultan insuficientes para identificar cuellos de botella o rastrear problemas a lo largo de múltiples servicios. Este trabajo de fin de máster propone la implementación de una solución integral de observabilidad, que combine monitoreo, gestión de logs y trazabilidad distribuida, utilizando herramientas estándar del ecosistema DevOps. La propuesta integra Prometheus y Grafana para el monitoreo, ELK Stack (Elasticsearch, Logstash y Kibana) para el procesamiento de logs, y OpenTelemetry junto con Jaeger para la trazabilidad de peticiones, todo desplegado en contenedores Docker y orquestado con Kubernetes. Se desarrollará una arquitectura de microservicios utilizando Spring Boot y se automatizará el despliegue mediante Helm y Ansible. El objetivo es evaluar la eficacia del sistema para detectar fallos y mejorar la visibilidad del comportamiento de los servicios. Como resultado, se espera ofrecer una solución práctica, reproducible y adaptable para equipos DevOps, que contribuya a mejorar la calidad del software y la capacidad de respuesta ante incidencias en entornos distribuidos.
\par\vspace{0.25cm}
\textbf{Palabras clave: } \keywordsESv \par
\end{abstract}


\pagebreak
%Ingles
\begin{abstract}
Abstract
\par
\par\vspace{0.25cm}
Modern microservices-based architectures introduce significant complexity and dynamism, making error detection, analysis, and resolution challenging. Traditional monitoring tools are often insufficient to identify bottlenecks or trace issues across distributed services. This thesis proposes the implementation of a comprehensive observability solution integrating monitoring, logging, and distributed tracing, using standard tools from the DevOps ecosystem. The proposed stack includes Prometheus and Grafana for metrics monitoring, the ELK Stack (Elasticsearch, Logstash, and Kibana) for centralized logging, and OpenTelemetry with Jaeger for request tracing. All components are deployed in Docker containers and orchestrated with Kubernetes to replicate a realistic production-like environment. A simple microservices architecture will be developed using Spring Boot, and deployment automation will be achieved through Helm and Ansible. The objective is to evaluate the effectiveness of the observability system in detecting failures, identifying bottlenecks, and improving traceability. The outcome is expected to provide a practical, replicable, and cloud-ready solution that helps DevOps teams enhance software quality and reduce response time to incidents in distributed environments.
\par\vspace{0.25cm}
\centering\textbf{Keywords: } \keywordsv \par
\end{abstract}
